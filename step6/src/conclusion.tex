\section{Conclusions}

In conclusion, the address generator created in step six worked flawlessly; the
timing of each bit was correct with respect to the clock produced in step five,
as well as relative to each other bit.  By using a JK flip-flop clocked by
the most significant bit produced by the 8-bit counters, a seventeenth bit was
produced.  While students could have simply added a third counter clocked by
that same source, this would have produced seven extra useless bits and would
have added unneeded complexity to the already complex system.

By using two \ttt{AND} gates to delay the clock signal into the JK flip-flop
controlling the read/write state of the system, students were able to ensure
that each generated address was stabilized before reading or writing occurred.
This~\SI{20}{\nano\second} delay prevented data corruption in the RAM, but was
short enough to not significantly affect the voice recorder's functionality.
